%\usepackage[german]{babel}
\usepackage[utf8]{inputenc}
\usepackage[a6paper,landscape]{geometry}
\usepackage{ifthen} % ifthenelse command
\usepackage{xifthen} % provides \isempty
\usepackage{amssymb}
\usepackage{xcolor}
\usepackage{graphicx} % for including pictures
\usepackage{xstring} %string operations
\usepackage{pgffor} %for-loop
\usepackage{amsmath} %for compoundfractions

\usepackage{units} % for nice units and unit-fractions

\usepackage{pifont} %define checkmark and xmark
\newcommand{\cmark}{\ding{51}}%
\newcommand{\xmark}{\ding{55}}%

\pagestyle{empty} % no page numbers
\setlength\parindent{0pt} %no indent

%%%%%%%%%%%%%%%%%%%%%%%%%%%%%%%%%%%%%%%%%%%%%%%%%%%%%%%%%%%%%%
%%%
%%% define search tags - edit to add more tags
%%%
% 4 Kategorien von Tags: 
% Kürzel, basales Jahr, Topic, und Schwierigkeitsgrad
%
% Um eine Kategorie zu erweitern, einfach unten in der Liste
% hinzufügen. Reihenfolge spielt keine Rolle.
% BEACHTE!! Nach dem letzten Eintrag in der Liste unten muss 
% ebenfalls ein Komma folgen.

% Liste: basale Phase
\newcommand{\basaltype}
{
basal0, %was sie beim Eintritt Gym1 wissen sollten
basal1.1, %was sie beim Test nach den Frühlingsferien wissen sollten (inkl. basal0)
basal1.2, %was sie bis Ende Gym1 zusätzlich zu basal 1.1 wissen sollten
}

% Liste: different topics
\newcommand{\topic}
{
Algebra, %basal 0 / 1.1 / 1.2 
Arithmetik, %basal 0 / 1.1 
Bruchterme, %basal 0 / 1.1 
Faktorisieren, %basal 0 / 1.1 
Funktionen, %basal 1.1 /1.2
Geometrie, %basal 0 / 1.2
Gleichungen, %basal 0 / 1.1 / 1.2
Grundoperationen, %basal 0 / 1.1
Koordinatensystem, %basal 0 / 1.1 / 1.2
Lineares, %basal 1.1 / 1.2
Mathematisieren, %basal 0 / 1.1 / 1.2
Quadratisches, %basal 1.1 / 1.2
Sachrechnen, %basal 0 / 1.1 
Wurzelterme, %basal 0 / 1.1 / 1.2
}

% Liste: degree of diffculty
\newcommand{\difficulty}
{
einfach, %war simple
schwerer, %war einfach
}
%%%%%%%%%%%%%%%%%%%%%%%%%%%%%%%%%%%%%%%%%%%%%%%%%%%%%%%%%%%%

%%%
%%% 'question' interface
%%%
\newcommand{\question}[1]{ 
\\ #1
}

%%%
%%% 'solution' interface
%%%
\newcommand{\solution}[1]{
%\vfill\hfill $\rightarrow$ Lösung: #1
\vfill\color{red} #1
}

%%%
%%% 'extract' command
%%%
\newcommand{\extract}[2]{
    \StrDel{#1}{,}[\mytempinput]% delete commas from string
    \StrLen{#1}[\len]% if empty string, add dummy character
    \ifthenelse{\equal{\len}{0}}{\let\mytempinput=a}{\ignorespaces}
    \foreach \tag in #2 {% loop through the list of expressions
        \IfSubStr{\mytempinput}{\tag}{%
            \StrDel{\mytempinput}{\tag}[\mytempinput]%
            \tag\ %if found in string, print expression
        }{%
            \ignorespaces % otherwise do nothing
        }%
        \global\let\mytempinput=\mytempinput% export outside
    }%
    %
    \StrLen{\mytempinput}[\len]%
    \ifthenelse{\equal{\len}{0}}{\ignorespaces}{\xmark}%
}

%%%
%%% 'add' interface 
%%%
\newenvironment{Add}[4]
{
    \hspace*{\fill} 
    {\it \tiny Tags found:
    % extract Kürzel - make cross if empty 
    \ifthenelse {\isempty{#1}} {\xmark}{#1} $|$%
    % extract basal type - make cross if empty of strange string
    \extract{#2}{\basaltype} $|$%
    % extract topics - make cross if empty of strange string
    \extract{#3}{\topic} $|$%
    % extract degree of difficulty - make cross if empty of strange string
    \extract{#4}{\difficulty}%
    }
    \\
    \noindent\rule{\textwidth}{1pt}% horizontal line
}
    %add question and solution
{
    \newpage
}
